\section{Spécifications Externes}
%\subsection{Besoins fonctionnels}
%\subsubsection{Inscription}
%Un utilisateur pourra s'inscrire grâce à un login et un mot de passe.
%\subsubsection{Connexion}
%Un utilisateur pourra se connecter à l'application via un login et un mot de passe.
%\subsubsection{Choix mode de jeu}
%Un utilisateur pourra créer un jeu avec plusieurs paramètres :
%\begin{itemize}
%	\item Nombre de joueurs humains ;
%	\item Nombre d'IA .
%\end{itemize}
%Dans le cas de la création d'un match avec plusieurs joueurs humains, une invitation leur sera envoyée.
%
%\subsubsection{Le jeu}
%\subsection{Contraintes fonctionnelles}
%
%Le nombre de joueurs est 8.
\subsection{Spécifications fonctionnelles}
\subsubsection{Fonctionnalité 1 : Gestion des utilisateurs}
Il y aura deux types d'utilisateurs dans un jeu : joueur \textbf{Admin} et joueur \textbf{Simple}. 
\subsubsection{Fonctionnalité 2 : Gestion du jeu}
Il y aura deux phases de jeu : \textbf{en attente} et \textbf{en cours}. Un jeu \textbf{en attente} est un jeu créé mais pas encore commencé. Un jeu \textbf{en cours} est un jeu déjà lancé par son joueur \textbf{Admin}. 
\subsubsection{Fonctionnalité  3 : Créer un nouveau jeu}
L'utilisateur sera capable de créer un nouveau jeu en cliquant sur le bouton \textbf{Créer un nouveau jeu} sur l'interface de lancement de partie. L'utilisateur qui crée le jeu est un joueur \textbf{Admin}. \textcolor{red}{Un joueur \textbf{Admin} aura la possibilité de choisir les paramètres du jeu. Une fois que le jeu est créé, il est en phase \textbf{en attente}.}
\subsubsection{Fonctionnalité  4 : S'inscrire dans un jeu existant}
L'utilisateur ne sera capable de s'inscrire que dans un jeu en phase \textbf{en attente}. Un utilisateur qui s'inscrit dans un jeu existant est un joueur \textbf{Simple}.
\subsubsection{Fonctionnalité 5 : Lancer un jeu}
Un joueur sera capable de lancer un jeu si et seulement s'il est un joueur \textbf{Admin} et qu'il y a au moins \textcolor{red}{un joueur \textbf{simple}} dans ce jeu. Une fois que le jeu est lancé, les autres ne peuvent plus s'inscrire dans le jeu.

\subsubsection{Fonctionnalité 6 : Afficher la page de jeu}
Une fois que le joueur \textbf{Admin} a lancé un jeu, la page de jeu sera affichée, elle contiendra les joueurs du jeu représentés par des avatars. La position d'un joueur au début du jeu est aléatoire. 

\subsubsection{Fonctionnalité 7 : Gestion d'une partie}
L'utilisateur sera capable de contrôler une IA par les touches directionnelles. La touche \textbf{Espace} sera utilisée pour déposer une bombe sur le plateau. Un joueur ne peut que déposer 10 bombes par jeu.

\subsubsection{Fonctionnalité 8 : Gestion d'une bombe}
Une fois qu'une bombe est déposée, un indicateur de temps est lancé pour que la bombe explose après 4 secondes. Les joueurs qui sont dans la même colonne ou la même ligne seront morts. Un joueur mort ne sera plus affiché sur le plateau et une interface \textbf{Choisir/Créer un jeu} est affiché automatiquement.
\subsubsection{Fonctionnalité 9 : La fin d'un jeu}
Le jeu se termine automatiquement s'il reste au plus un joueur dans la partie ou que tous les joueurs n'ont plus de bombe. Une interface \textbf{Winner} sera affichée automatiquement pour chaque joueur qui reste dans la partie et puis chaque joueur sera redirigé vers l'interface \textbf{Choisir/Créer un jeu} après 3 secondes.
\subsubsection{Fonctionnalité 10 : Gestion des interfaces}
Il aura \textcolor{red}{quatre} interfaces :l'interface \textbf{Login}, l'interface \textbf{Choisir/Créer un jeu}, l'interface \textbf{Jouer} et l'interface \textbf{Winner}.
\begin{enumerate}
	\item \textcolor{red}{L'interface \textbf{Connexion} contient un champs login à remplir pour identifier le joueur};
	\item L'interface \textbf{Choisir/Créer un jeu} contient une liste des jeux existants (en attente/ en cours ) et une bouton \textbf{Créer un nouveau jeu} ;
	\item L'interface \textbf{Jouer} contient le plateau de jeu \textcolor{red}{et le nombre des bombes de chaque joueur inscrit};
	\item L'interface \textbf{Winner} ne contient qu'une félicitation textuelle, et une redirection automatique vers l'interface \textbf{Choisir/Créer un jeu};
\end{enumerate}

\subsection{Contraintes fonctionnelles}
Le nombre de joueurs dans un jeu est de 2 à \textcolor{red}{5}. Un jeu ne peut être lancé que s'il y a au moins un joueur \textbf{Simple} d'inscrit dans la partie.
